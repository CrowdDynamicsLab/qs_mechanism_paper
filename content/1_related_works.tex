\section{Related Work}
\label{sec:relatedWorks}
In this section, we describe related works regarding QS and the quadratic mechanism embedded within. We then describe related work in force choice surveying techniques that follow a linear constraint.

\subsection{Quadratic Surveys and the Quadratic Mechanism}
Quadratic Surveys (QS) utilize the quadratic mechanism, where participants express their preferences by ``purchasing'' approval or disapproval votes given a fixed budget. The cost of each additional vote increases quadratically, discouraging extreme responses and encouraging participants to carefully distribute their votes based on relative preference strength. Participants can use different number of positive and negative votes to demonstrate relative preferences across options. This mechanism allows survey designers to aggregate individual collective preferences by summing individual preference intensities across all participants.

Formally, if a survey presents $K$ options and each respondent receives a budget $B$, a participant can allocate $n_k$ votes to an option $k$, with the cost of votes following a quadratic function: $c_k = n_k^2$, where $n_k \in \mathbb{Z}$. The votes can be positive or negative to indicate the strength of support for an option. Respondents must ensure that their total expenditure does not exceed their budget: $\sum_{k=1}^{K} c_k \leq B$. The collective preference for each option is then determined by summing the votes across all participants: $\sum_{i=1}^{S} n_{i,k}$, where $S$ is the number of respondents and $n_{i,k}$ represents the votes allocated by participant $i$ to option $k$.

The quadratic mechanism originates from economic theory, particularly in incentive-compatible mechanisms for public goods allocation~\cite{grovesOptimalAllocationPublic1977}. It gained prominence through \textbf{Quadratic Voting (QV)}, which addresses the ``tyranny of the majority'' by allowing individuals to express preference intensity rather than making a binary choice~\cite{posner2018radical}. Unlike voting, QS is not intended for decision-making but rather for eliciting preferences that inform decision entities or the public~\cite{chengOrganizeThenVote2025}.

Empirical studies have explored QS across different settings, including laboratory experiments~\cite{chengCanShowWhat2021,quarfoot2017quadratic}, public policy polls~\cite{cavaille2024cares, hollandDistributiveImpactsSupport2022}, and education research~\cite{naylor2017first}. These studies have demonstrated that QS enables the simultaneous elicitation of rankings and ratings---an advantage over traditional survey methods~\cite{chengCanShowWhat2021}. Additionally, QS mitigates extreme response biases, even in polarized topics, while capturing richer preference data than conventional Likert scales~\cite{quarfoot2017quadratic, cavaille2024cares, chengCanShowWhat2021, naylor2017first}. Recent work has further validated QS by showing alignment between reported preferences and observed in-lab behaviors, suggesting that QS outperforms Likert scales in capturing true preferences.

However, QS imposes cognitive demands on participants. Empirical studies indicate that respondents experience medium to high cognitive load, particularly when evaluating many options~\cite{cavaille2024cares, chengOrganizeThenVote2025}. As a result, researchers have proposed alternative designs to simplify QS, such as replacing the quadratic cost structure with a linear one. Despite theoretical discussions, no empirical work has systematically evaluated the trade-offs between quadratic and linear cost structures to date.

\subsection{Linear constraint-based Collective Decision Making Mechanisms}
\label{sec:related_works_force_choice}
While QS's quadratic cost structure is novel within survey design, the core idea of constraining participants to operate within a fixed budget is well established in fields such as marketing research, psychology, and political science. Existing tools intentionally impose trade-offs where individuals are asked to allocate a limited number of points, tokens, or monetary units across multiple options. In this subsection, we examine two commonly known techniques that employ such linear constraints: \textit{Constant Sum Surveys} and \textit{Knapsack Voting}.

\subsubsection{Constant Sum surveys}
Constant Sum Surveys (CSS) have existed since the 1950s~\cite{Malhotra_Naresh_K_2012, smithBasicMarketingResearch2013, Donald_R_Cooper2013-03-05}, originating as two-choice splits of 100 points~\cite{metfesselProposalQuantitativeReporting1947} and later generalized to multi-option contexts~\cite{zhuSelfestimationWeightParameter1991, harwoodUnderstandingImplicitExplicit2019}. In CSS, participants are given a fixed budget of points (often 100) to distribute among $K$ options, reflecting each option’s relative importance or appeal. Although commercial survey platforms differ in their exact implementation~\cite{qualtricsConstantSumQuestion2025,surveysparrowWhatConstantSum2025,lorraineConstantSumQuestion2022}, the underlying principle is straightforward: respondents cannot exceed the total points allocated to them.

Studies show CSS elicits both ranking and rating information, making it useful in marketing or political surveys~\cite{collewetPreferenceEstimationPoint2023}. Validation against behavioral measures is mixed: CSS often aligns with pairwise comparisons~\cite{dudekValidityPointAssignmentProcedure1957} but can diverge from revealed preferences such as willingness to pay (WTP)~\cite{louviereComparisonImportanceWeights2008}. Despite such nuances, CSS remains popular for capturing intensities within a linear budget constraint.

In this study, LS lies close to CSS with three minor differences. First, CSS does not typically allow a negative point allocation. Second, many CSS implementations require participants to consume the full budget. Last, CSS does not frame the process as `vote buying.'  Mathematically, negative votes is equivalent of having an additional set of disagreement options; full consumption of the credit refers to an additional dummy option. Thus, individuals could behave differently under these framings~\cite{shahScarcityFramesValue2015, kahnemanProspectTheoryAnalysis1979}. Hence we do not attribute LS as CSS in this study but included relevant literature since it closely resembles LS.

\subsubsection{Knapsack Voting and participatory budgeting}
Knapsack Voting (KV) is another forced-choice surveying technique developed for participatory budgeting, a process through which community members express their preferences for how public resources should be allocated~\cite{goelKnapsackVotingParticipatory2019, goel2016budget}. In KV, each individual is given a fixed budget to select from a set of options, each associated with a specific cost. Participants may choose any combination of projects, provided the total cost does not exceed their allotted budget. This approach requires participants to contribute predefined `chunks' of budget following a linear relationship, which we do not explore in this study.

% old tex:
% The reliability and validity of CSS have been examined in limited contexts. CSS originated from pairwise comparative studies~\cite{metfesselProposalQuantitativeReporting1947}, where participants split 100 points between two choices. Subsequent research extended CSS to multi-option settings, showing comparable outcomes to pairwise comparisons~\cite{collewetPreferenceEstimationPoint2023}.  Some studies validate its use in measuring physical object differences (e.g., weight perception)~\cite{dudekValidityPointAssignmentProcedure1957}, while others assess its ability to predict real-world behaviors, such as consumer purchase decisions in marketing~\cite{collewetPreferenceEstimationPoint2023}. However, these studies primarily evaluate CSS’s ordinal accuracy rather than its capacity to measure intensity. Research comparing CSS with conjoint analysis, willingness-to-pay (WTP) measures, and other survey techniques has produced mixed findings. While CSS and conjoint analysis often yield similar stated preferences, discrepancies emerge when compared to behavioral measures like WTP~\cite{louviereComparisonImportanceWeights2008}. This reflects broader concerns in behavioral economics regarding the divergence between stated and revealed preferences~\cite{collewetPreferenceEstimationPoint2023, louviereComparisonImportanceWeights2008}. Despite these uncertainties, CSS remains widely used in marketing, political science, and psychometrics because it can elicit both rankings and ratings.