\section{Related Work}
\label{sec:relatedWorks}
In this section, we describe related works regarding QS and the quadratic mechanism embedded within. We also describe constant sum surveys which is the closest surveying mechanism that mirrors linear voting.

\subsection{Quadratic Surveys and the Quadratic Mechanism}
Quadratic Surveys (QS) utilize the quadratic mechanism, where participants express their preferences by ``purchasing'' approval or disapproval votes given a fixed budget. The cost of each additional vote increases quadratically, discouraging extreme responses and encouraging participants to carefully distribute their votes based on relative preference strength. Participants can use different number of positive and negative votes to demonstrate relative preferences across options. This mechanism allows survey designers to aggregate individual collective preferences by summing individual preference intensities across all participants.

Formally, if a survey presents $K$ options and each respondent receives a budget $B$, a participant can allocate $n_k$ votes to an option $k$, with the cost of votes following a quadratic function: $c_k = n_k^2$, where $n_k \in \mathbb{Z}$. Respondents must ensure that their total expenditure does not exceed their budget: $\sum_{k=1}^{K} c_k \leq B$. The collective preference for each option is then determined by summing the votes across all participants: $\sum_{i=1}^{S} n_{i,k}$, where $S$ is the number of respondents and $n_{i,k}$ represents the votes allocated by participant $i$ to option $k$.

The quadratic mechanism originates from economic theory, particularly in incentive-compatible mechanisms for public goods allocation~\cite{grovesOptimalAllocationPublic1977}. It gained prominence through \textbf{Quadratic Voting (QV)}, which addresses the ``tyranny of the majority'' by allowing individuals to express preference intensity rather than making a binary choice~\cite{posner2018radical}. Unlike voting, QS is not intended for decision-making but rather for eliciting preferences that inform decision entities or the public~\cite{chi}.

Empirical studies have explored QS across different settings, including laboratory experiments~\cite{}, mini-public deliberations~\cite{}, and cross-disciplinary applications. These studies have demonstrated that QS enables the simultaneous elicitation of rankings and ratings—an advantage over traditional survey methods~\cite{chengCanShowWhat2021}. Additionally, QS mitigates extreme response biases, even in polarized topics, while capturing richer preference data than conventional Likert scales~\cite{quarfoot2017quadratic, cavaille2024cares, chengCanShowWhat2021, naylor2017first}. Recent work has further validated QS by showing alignment between reported preferences and observed in-lab behaviors, suggesting that QS outperforms Likert scales in capturing true preferences.

However, QS imposes cognitive demands on participants. Empirical studies indicate that respondents experience medium to high cognitive load, particularly when evaluating many options~\cite{cavaille2024cares, chengCanShowWhat2021}. As a result, researchers have proposed alternative designs to simplify QS, such as replacing the quadratic cost structure with a linear one. Despite theoretical discussions, no empirical work has systematically evaluated the trade-offs between quadratic and linear cost structures to date.

\subsection{Constraint-based collective decision making mechanism}
\label{sec:related_works_css}
Linear voting closely resembles Constant Sum Survey (CSS), also known as chip allocation or point allocation surveys, which are widely used in marketing research~\cite{Malhotra_Naresh_K_2012, smithBasicMarketingResearch2013, Donald_R_Cooper2013-03-05, toepoelSmileysStarsHearts2019, hauserIntensityMeasuresConsumer1980} and situations where there are multiple choices~\cite{zhuSelfestimationWeightParameter1991, harwoodUnderstandingImplicitExplicit2019, hanPatientsPerspectivePsychiatric2024}.

While there are no one standard CSS mechanism used across commercial surveying tools~\cite{qualtricsConstantSumQuestion2025, surveysparrowWhatConstantSum2025, lorraineConstantSumQuestion2022} which involves participants distributing a fixed number of points across multiple options, directly indicating their relative preferences, reflecting a linear fashion. CSS among these platforms differ in the the freedom for survey designers to indicate the total number to be distributed, and whether the question allows unused points.

Linear Survey, similar to CSS in the way degree of preferences remains a linear relationship with the cost from their given budget, it introduced slightly more freedom then CSS. First, it does not force survey respondents to use up all their budget. Second, participants are allowed to express disagree votes. And third, agree votes will cancel out disagree votes for the same option across all participants when it comes to aggregating collective outcomes.

Mathematically, unused credits could be translated into an additional option, with the label -- no opinion -- in a CSS survey; the disagree votes can be either seen as two options on a CSS survey, one positive and one negative. However, a direct distribution in CSS and the indirect of QS might have framed participants into a scarcity mindset~\cite{shahScarcityFramesValue2015} or prospect theory~\cite{kahnemanProspectTheoryAnalysis1979} that makes survey respondents react to the two instruments~\textit{differently}. Thus, this paper will use the two terms to represent them.

The reliability and validity of CSS have been examined in limited contexts. CSS originated from pairwise comparative studies~\cite{metfesselProposalQuantitativeReporting1947}, where participants split 100 points between two choices. Subsequent research extended CSS to multi-option settings, showing comparable outcomes to pairwise comparisons~\cite{collewetPreferenceEstimationPoint2023}.  Some studies validate its use in measuring physical object differences (e.g., weight perception)~\cite{dudekValidityPointAssignmentProcedure1957}, while others assess its ability to predict real-world behaviors, such as consumer purchase decisions in marketing~\cite{collewetPreferenceEstimationPoint2023}. However, these studies primarily evaluate CSS’s ordinal accuracy rather than its capacity to measure intensity. Research comparing CSS with conjoint analysis, willingness-to-pay (WTP) measures, and other survey techniques has produced mixed findings. While CSS and conjoint analysis often yield similar stated preferences, discrepancies emerge when compared to behavioral measures like WTP~\cite{louviereComparisonImportanceWeights2008}. This reflects broader concerns in behavioral economics regarding the divergence between stated and revealed preferences~\cite{collewetPreferenceEstimationPoint2023, louviereComparisonImportanceWeights2008}.

Despite these uncertainties, CSS remains widely used in marketing, political science, and psychometrics because it can elicit both rankings and ratings. A key difference between QS and CSS is that QS permits participants to spend less than their total budget, whereas many CSS implementations require full allocation to maintain the pairwise nature of the original method. Additionally, CSS does not frame the allocation process as "purchasing votes," avoiding potential biases introduced by transactional framing. Given these minor distinctions, we consider QS with a linear cost function to be functionally equivalent to CSS, since removing quadratic costs eliminates QS’s key differentiating feature.
