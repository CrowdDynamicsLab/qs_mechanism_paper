\section{Related Work}
\label{sec:relatedWorks}

This study examines the differences between Quadratic Surveys (QS) and Constant Sum Surveys (CSS) by comparing their ability to elicit preferences accurately. In this section, we outline prior work on the quadratic mechanism, the design and applications of QS, and the properties of CSS as a baseline comparison.

\subsection{Quadratic Surveys and the Quadratic Mechanism}
Quadratic Surveys (QS) utilize the quadratic mechanism, where participants express their preferences by "purchasing" votes using a fixed budget. The cost of each additional vote increases quadratically, discouraging extreme responses and encouraging participants to carefully distribute their votes based on relative preference strength. This mechanism allows survey designers to aggregate collective preferences by summing individual preference intensities across all participants.

Formally, if a survey presents $K$ options and each respondent receives a budget $B$, a participant can allocate $n_k$ votes to an option $k$, with the cost of votes following a quadratic function: $c_k = n_k^2$, where $n_k \in \mathbb{Z}$. Respondents must ensure that their total expenditure does not exceed their budget: $\sum_{k=1}^{K} c_k \leq B$. The collective preference for each option is then determined by summing the votes across all participants: $\sum_{i=1}^{S} n_{i,k}$, where $S$ is the number of respondents and $n_{i,k}$ represents the votes allocated by participant $i$ to option $k$.

The quadratic mechanism originates from economic theory, particularly in incentive-compatible mechanisms for public goods allocation~\cite{grovesOptimalAllocationPublic1977}. It gained prominence through \textbf{Quadratic Voting (QV)}, which addresses the "tyranny of the majority" by allowing individuals to express preference intensity rather than making a binary choice~\cite{posner2018radical}. Unlike voting, QS is not intended for decision-making but rather for eliciting preferences that inform decision entities or the public~\cite{chi}.

Empirical studies have explored QS across different settings, including laboratory experiments~\cite{}, mini-public deliberations~\cite{}, and cross-disciplinary applications. These studies have demonstrated that QS enables the simultaneous elicitation of rankings and ratings—an advantage over traditional survey methods~\cite{chengCanShowWhat2021}. Additionally, QS mitigates extreme response biases, even in polarized topics, while capturing richer preference data than conventional Likert scales~\cite{quarfoot2017quadratic, cavaille2024cares, chengCanShowWhat2021, naylor2017first}. Recent work has further validated QS by showing alignment between reported preferences and observed in-lab behaviors, suggesting that QS outperforms Likert scales in capturing true preferences.

However, QS imposes cognitive demands on participants. Empirical studies indicate that respondents experience medium to high cognitive load, particularly when evaluating many options~\cite{cavaille2024cares, chengCanShowWhat2021}. As a result, researchers have proposed alternative designs to simplify QS, such as replacing the quadratic cost structure with a linear one. Despite theoretical discussions, no empirical work has systematically evaluated the trade-offs between quadratic and linear cost structures to date.

\subsection{Constant Sum Method}
\label{sec:related_works_css}
A linear version of QS closely resembles the Constant Sum Survey (CSS), also known as chip allocation or point allocation surveys, which are widely used in marketing research~\cite{}. In CSS, participants distribute a fixed number of points across multiple options, directly indicating their relative preferences. Unlike QS, where the cost function transforms preferences non-linearly, CSS preserves a direct proportionality between allocation and preference strength.

While there is no standard point allocation threshold, CSS typically assigns respondents 100 points to distribute among a small set of options. The method originated from pairwise comparative studies~\cite{metfesselProposalQuantitativeReporting1947}, where participants split 100 points between two choices. Subsequent research extended CSS to multi-option settings, showing comparable outcomes to pairwise comparisons~\cite{}.

The reliability and validity of CSS have been examined in limited contexts. Some studies validate its use in measuring physical object differences (e.g., weight perception)~\cite{}, while others assess its ability to predict real-world behaviors, such as consumer purchase decisions in marketing~\cite{}. However, these studies primarily evaluate CSS’s ordinal accuracy rather than its capacity to measure intensity. Research comparing CSS with conjoint analysis, willingness-to-pay (WTP) measures, and other survey techniques has produced mixed findings. While CSS and conjoint analysis often yield similar stated preferences, discrepancies emerge when compared to behavioral measures like WTP~\cite{}. This reflects broader concerns in behavioral economics regarding the divergence between stated and revealed preferences~\cite{}.

Despite these uncertainties, CSS remains widely used in marketing, political science, and psychometrics because it can elicit both rankings and ratings. A key difference between QS and CSS is that QS permits participants to spend less than their total budget, whereas many CSS implementations require full allocation to maintain the pairwise nature of the original method. Additionally, CSS does not frame the allocation process as "purchasing votes," avoiding potential biases introduced by transactional framing. Given these minor distinctions, we consider QS with a linear cost function to be functionally equivalent to CSS, since removing quadratic costs eliminates QS’s key differentiating feature.
