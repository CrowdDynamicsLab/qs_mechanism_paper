\section{Related Work}
\label{sec:relatedWorks}
In this section, we describe related works regarding QS and the quadratic mechanism embedded within. We then describe related work in forced choice surveying techniques that follow a linear constraint.

\subsection{Quadratic Surveys and the Quadratic Mechanism}
QS uses a quadratic mechanism in which participants `purchase' approval or disapproval votes to express their preference within a fixed budget. Because vote cost increases quadratically, participants are discouraged from extreme responses and encouraged to allocate votes based on relative preference strength. Participants may assign varying numbers of positive and negative votes to reflect relative preferences. Survey designers compute collective preferences by summing votes for each option across participants.

Formally, a participant receives a QS with $K$ options and a budget $B$, and may allocate $n_k$ votes to each option $k$, with vote cost defined by a quadratic function: $c_k = n_k^2$, where $n_k \in \mathbb{Z}$. Votes may be positive or negative to express support or opposition. Respondents must ensure that their total expenditure does not exceed their budget: $\sum_{k=1}^{K} c_k \leq B$. The collective preference for each option is then determined by summing the votes across all participants: $\sum_{i=1}^{S} n_{i,k}$, where $S$ is the number of respondents and $n_{i,k}$ represents the votes allocated by participant $i$ to option $k$.

The quadratic mechanism originates from economic theory, particularly for public goods allocation~\cite{grovesOptimalAllocationPublic1977}. It gained prominence through \textbf{Quadratic Voting (QV)}, which addresses the ``tyranny of the majority'' by allowing individuals to express preference intensity rather than cast a binary vote~\cite{posner2018radical}. Unlike QV, which produces binding decisions, QS gathers opinions to inform decision-makers or the public~\cite{cavaille2024cares}. %cite CHI paper later

Empirical studies have evaluated QS in settings ranging from lab experiments~\cite{chengCanShowWhat2021,quarfoot2017quadratic} to policy polling~\cite{cavaille2024cares, hollandDistributiveImpactsSupport2022}, and education research~\cite{naylor2017first}. They show that QS elicits both rankings and ratings---an advantage over traditional survey methods~\cite{chengCanShowWhat2021}. QS also reduces  extreme response biases, even on polarized topics, and captures richer preference data than Likert scale surveys~\cite{quarfoot2017quadratic, cavaille2024cares, chengCanShowWhat2021, naylor2017first}. Recent studies have reported stronger alignment between QS-based stated preferences and observed behavior compared to Likert surveys~\cite{cavaille2024cares, chengCanShowWhat2021}.

QS requires higher cognitive demands to complete, with empirical studies showing that participants report medium to high cognitive load, especially when evaluating longer lists of options~\cite{cavaille2024cares}. While heightened cognitive load can lead to deeper engagement with survey options, prior survey research literature suggests it also drives down participation rates and increases dropout~\cite{brosnanCognitiveLoadReduction2021, galesicDropoutsWebEffects2006}. In response, researchers~\cite{cavaille2024cares} have proposed simplifying QS by replacing the quadratic cost with a linear one. Yet no empirical study has systematically examined the trade-offs between quadratic and linear cost structures.

\subsection{Linear Constraint-Based Collective Decision-Making Mechanisms}
\label{sec:related_works_force_choice}
While QS's quadratic cost structure is novel, the practice of imposing fixed budgets in surveys has a long history in marketing, psychology, and political science. These methods require participants to allocate a limited points, tokens, or money across options, forcing trade-offs. Among these,~\textit{Constant Sum Surveys} and~\textit{Knapsack Voting} (KV) are the most relevant comparisons to QS. Unlike other forced-choice techniques, such as MaxDiff~\cite{tsafarakisInvestigatingPreferencesIndividuals2019, schrammIncentiveAlignmentAnchored2024}, Best-Worst Scaling~\cite{louviereBestWorstScalingTheory2015}, or conjoint analysis~\cite{bagozziAdvancedMarketingResearch1994}, CSS and KV impose explicit linear resource constraints, making them conceptually closer to the QS and LS examined in this study.

\subsubsection{Constant Sum surveys}
CSS has existed since the 1950s~\cite{Malhotra_Naresh_K_2012, smithBasicMarketingResearch2013, Donald_R_Cooper2013-03-05}, originally designed as 100-point splits between two options~\cite{metfesselProposalQuantitativeReporting1947} and later extended to multiple-option settings~\cite{zhuSelfestimationWeightParameter1991, harwoodUnderstandingImplicitExplicit2019}. In CSS, participants receive a fixed point budget (often 100) to distribute across $K$ options, reflecting their relative perceived importance. Although survey platforms vary in implementation~\cite{qualtricsConstantSumQuestion2025, surveysparrowWhatConstantSum2025, lorraineConstantSumQuestion2022}, the core constraint remains: respondents must stay within their allotted budget.

Studies show CSS elicits both ranking and rating information, making it useful in domains such as marketing and political science~\cite{collewet2023preference}. Validation against behavioral measures is mixed: CSS often aligns with pairwise comparisons~\cite{dudekValidityPointAssignmentProcedure1957}, but can diverge from revealed preferences, as reflected in measures like willingness to pay (WTP)~\cite{louviereComparisonImportanceWeights2008}. Despite these differences, CSS remains popular for capturing preference intensities within a linear budget constraint.

LS closely resembles CSS but differs in three minor ways. First, CSS does not typically allow negative point allocations. Second, many CSS implementations require participants to exhaust the full budget. Last, CSS is typically not framed as a vote allocation process, unlike QS, which emphasizes 'vote buying' as part of its interface metaphor. While LS can be reformulated to match CSS mathematically, for example, by interpreting the negative votes as additional disagreement options on the survey, or residual budgets as a dummy option, differences in framing may lead to distinct participant behaviors~\cite{shahScarcityFramesValue2015, kahnemanProspectTheoryAnalysis1979}. Accordingly, we conservatively treat LS as distinct from CSS, though their structural similarities support methodological comparisons.

\subsubsection{Knapsack Voting and participatory budgeting}
KV is another forced-choice surveying technique developed for participatory budgeting, a process where community members express preferences for how public resources should be allocated~\cite{goelKnapsackVotingParticipatory2019, goel2016budget}. In KV, participants receive a fixed budget and select from options with predefined costs. Participants may choose any combination of options, as long as the total cost remains within budget. This approach requires participants to contribute predefined `chunks' of budget following a linear relationship, which we do not explore in this study, as QS options do not necessarily come with defined costs.

% old tex:
% The reliability and validity of CSS have been examined in limited contexts. CSS originated from pairwise comparative studies~\cite{metfesselProposalQuantitativeReporting1947}, where participants split 100 points between two choices. Subsequent research extended CSS to multi-option settings, showing comparable outcomes to pairwise comparisons~\cite{collewetPreferenceEstimationPoint2023}.  Some studies validate its use in measuring physical object differences (e.g., weight perception)~\cite{dudekValidityPointAssignmentProcedure1957}, while others assess its ability to predict real-world behaviors, such as consumer purchase decisions in marketing~\cite{collewetPreferenceEstimationPoint2023}. However, these studies primarily evaluate CSS's ordinal accuracy rather than its capacity to measure intensity. Research comparing CSS with conjoint analysis, willingness-to-pay (WTP) measures, and other survey techniques has produced mixed findings. While CSS and conjoint analysis often yield similar stated preferences, discrepancies emerge when compared to behavioral measures like WTP~\cite{louviereComparisonImportanceWeights2008}. This reflects broader concerns in behavioral economics regarding the divergence between stated and revealed preferences~\cite{collewetPreferenceEstimationPoint2023, louviereComparisonImportanceWeights2008}. Despite these uncertainties, CSS remains widely used in marketing, political science, and psychometrics because it can elicit both rankings and ratings.