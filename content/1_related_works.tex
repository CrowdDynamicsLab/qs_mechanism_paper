\section{Related Work}
\label{sec:relatedWorks}
In this section, we describe related works regarding QS and the quadratic mechanism embedded within. We then describe related work in force choice surveying techniques that follow a linear constraint.

\subsection{Quadratic Surveys and the Quadratic Mechanism}
QS use a quadratic mechanism in which participants `purchase' approval or disapproval votes to express their preference within a fixed budget. Because vote cost increases quadratically, participants are discouraged from extreme responses and encouraged to allocate votes based on relative preference strength. Participants may assign varying numbers of positive and negative votes to reflect relative preferences. Survey designers compute collective preferences by summing votes for each option across participants.

Formally, a participant receives a QS with $K$ options and a budget $B$, and may allocate $n_k$ votes to each option $k$, with vote cost defined by a quadratic function: $c_k = n_k^2$, where $n_k \in \mathbb{Z}$. Votes may be positive or negative to express support or opposition. Respondents must ensure that their total expenditure does not exceed their budget: $\sum_{k=1}^{K} c_k \leq B$. The collective preference for each option is then determined by summing the votes across all participants: $\sum_{i=1}^{S} n_{i,k}$, where $S$ is the number of respondents and $n_{i,k}$ represents the votes allocated by participant $i$ to option $k$.

The quadratic mechanism originates from economic theory, particularly for public goods allocation~\cite{grovesOptimalAllocationPublic1977}. It gained prominence through \textbf{Quadratic Voting (QV)}, which addresses the ``tyranny of the majority'' by allowing individuals express preference intensity rather than cast a binary vote~\cite{posner2018radical}. Unlike voting, QS is not intended for direct results binding decision-making but for eliciting preferences to inform decision-makers or the public~\cite{chengOrganizeThenVote2025}.

Empirical studies have evaluated QS in settings ranging from lab experiments~\cite{chengCanShowWhat2021,quarfoot2017quadratic} to policy polling~\cite{cavaille2024cares, hollandDistributiveImpactsSupport2022}, and education research~\cite{naylor2017first}. They show that QS elicits both rankings and ratings---an advantage over traditional survey methods~\cite{chengCanShowWhat2021}. QS also reduces  extreme response biases, even on polarized topics, and captures richer preference data than Likert scale surveys~\cite{quarfoot2017quadratic, cavaille2024cares, chengCanShowWhat2021, naylor2017first}. Recent studies~\cite{cavaille2024cares, chengCanShowWhat2021} validates QS by showing stronger alignment between stated and observed preferences than with Likert scales.

However, QS imposes cognitive demands on participants. Empirical studies show participants report medium to high cognitive load, especially when they evaluate longer lists of options~\cite{cavaille2024cares, chengOrganizeThenVote2025}. In response, researchers have proposed simplifying QS by replacing the quadratic cost with a linear one. Yet no empirical study has systematically examined the trade-offs between quadratic and linear cost structures.

\subsection{Linear constraint-based Collective Decision Making Mechanisms}
\label{sec:related_works_force_choice}
While QS's quadratic cost structure is novel, imposing fixed budgets in surveys is a long-standing method in marketing, psychology, and political science. Participants must distribute limited points, tokens, or money across options, forcing explicit trade-offs. We now examine two common techniques that rely on linear constraints: \textit{Constant Sum Surveys} and \textit{Knapsack Voting}.

\subsubsection{Constant Sum surveys}
Constant Sum Surveys (CSS) have existed since the 1950s~\cite{Malhotra_Naresh_K_2012, smithBasicMarketingResearch2013, Donald_R_Cooper2013-03-05}, originally designed as 100-point splits between two options~\cite{metfesselProposalQuantitativeReporting1947} and later extended to multiple-option settings~\cite{zhuSelfestimationWeightParameter1991, harwoodUnderstandingImplicitExplicit2019}.In CSS, participants receive a fixed point budget (often 100) to distribute across $K$ options, reflecting their relative perceived importance. Although survey platforms vary in implementation~\cite{qualtricsConstantSumQuestion2025, surveysparrowWhatConstantSum2025, lorraineConstantSumQuestion2022}, the core constraint remains: respondents must not exceed their allocated total.

Studies show CSS elicits both ranking and rating information, making it useful in domains such as marketing and political science~\cite{collewetPreferenceEstimationPoint2023}. Validation against behavioral measures is mixed: CSS often aligns with pairwise comparisons~\cite{dudekValidityPointAssignmentProcedure1957}, but can diverge from revealed preferences, as reflected in measures like willingness to pay (WTP)~\cite{louviereComparisonImportanceWeights2008}. Despite these nuances, CSS remains popular for capturing preference intensities within a linear budget constraint.

LS closely resembles CSS but differs in three minor ways. First, CSS does not typically allow a negative point allocation. Second, many CSS implementations require participants to consume the full budget. Last, CSS does not frame the process as `vote buying.' Mathematically, allowing negative votes is equivalent to adding disagreement options; full budget use can imply a dummy option for residual points. Thus, individuals could behave differently under these framings~\cite{shahScarcityFramesValue2015, kahnemanProspectTheoryAnalysis1979}. Hence we do not attribute LS as CSS in this study but included relevant literature since it closely resembles LS.

\subsubsection{Knapsack Voting and participatory budgeting}
Knapsack Voting (KV) is another forced-choice surveying technique developed for participatory budgeting, a process where community members express preferences for how public resources should be allocated~\cite{goelKnapsackVotingParticipatory2019, goel2016budget}. In KV, participants receive a fixed budget and select from options with predefined costs. Participants may choose any combination of projects, as long as the total cost remains within budget. This approach requires participants to contribute predefined `chunks' of budget following a linear relationship, which we do not explore in this study, as QS options do not necessarily come with defined costs.

% old tex:
% The reliability and validity of CSS have been examined in limited contexts. CSS originated from pairwise comparative studies~\cite{metfesselProposalQuantitativeReporting1947}, where participants split 100 points between two choices. Subsequent research extended CSS to multi-option settings, showing comparable outcomes to pairwise comparisons~\cite{collewetPreferenceEstimationPoint2023}.  Some studies validate its use in measuring physical object differences (e.g., weight perception)~\cite{dudekValidityPointAssignmentProcedure1957}, while others assess its ability to predict real-world behaviors, such as consumer purchase decisions in marketing~\cite{collewetPreferenceEstimationPoint2023}. However, these studies primarily evaluate CSS's ordinal accuracy rather than its capacity to measure intensity. Research comparing CSS with conjoint analysis, willingness-to-pay (WTP) measures, and other survey techniques has produced mixed findings. While CSS and conjoint analysis often yield similar stated preferences, discrepancies emerge when compared to behavioral measures like WTP~\cite{louviereComparisonImportanceWeights2008}. This reflects broader concerns in behavioral economics regarding the divergence between stated and revealed preferences~\cite{collewetPreferenceEstimationPoint2023, louviereComparisonImportanceWeights2008}. Despite these uncertainties, CSS remains widely used in marketing, political science, and psychometrics because it can elicit both rankings and ratings.