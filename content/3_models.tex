\section{Modeling for Pairwise Ranking and Preference Intensity Analyses}
\label{sec:quantitative_measures}
Two recent empirical studies have evaluated whether elicited survey responses align with participant behavior, using outcomes such as charitable donations~\cite{chengCanShowWhat2021, cavaille2024cares} or letter-writing effort~\cite{cavaille2024cares} as behavioral proxies for underlying preferences. One approach used Bayesian cosine similarity to compare high-dimensional response vectors with behavioral outcomes~\cite{chengCanShowWhat2021}; another applied linear regression to estimate the gap between stated and revealed preferences~\cite{cavaille2024cares}.

However, cosine similarity poses interpretability challenges: vectors with identical pairwise rankings can still be judged dissimilar, while near-aligned vectors may reflect contradictory preferences. Moreover, distinct behavioral measures (e.g., donations vs. letter writing) complicate comparisons of pairwise preferences across options within the same participant.

To address these limitations, we evaluate survey instruments based on their ability to recover (1) \textit{pairwise preference rankings} and (2) \textit{preference intensity differences} between options, within participants. This dual evaluation draws from methods used in studies of point-allocation and forced-choice surveys~\cite{collewet2023preference}, and allows us to separately assess ordinal and interval-level performance. 

We employ Bayesian modeling in both cases to support transparent assumptions, account for uncertainty, and enable interpretation beyond binary significance thresholds~\cite{mcelreath2018statistical, kay2016researcher}. The two models are described below.

\subsection{Pairwise Ranking Model}
\label{sec:ordinal_measures}
Our first analysis evaluates how well different survey instruments capture pairwise rankings that align with those inferred from actual donation amounts. We model the binary observation ($y_i$) of whether participant $i$'s pairwise ranking expressed via the survey instrument matches that from the donation results for a given societal issue pair as a Bernoulli distribution in \Cref{eq:ordinal_model_overall}:

\begin{equation}
    \label{eq:ordinal_model_overall}
    y_i \sim \text{Bernoulli}(\theta_i)
\end{equation}

The alignment probability, $\theta_i$, is defined via a logit link function (Eq.~\ref{eq:ordinal_model_logit}) and modeled as a function of several experimental variables:

\begin{equation}
    \label{eq:ordinal_model_logit}
    \text{logit}(\theta_i) = \alpha + \beta_c[C_i] + \beta_o[O_i] + \beta_p[P_i] + \beta_t[T_{1i}] + \beta_t[T_{2i}]
\end{equation}

The variables represent experimental conditions and relevant controls. Specifically, $C_i$ denotes the survey instrument, spanning eight conditions (see \Cref{tbl:experiment_cond}\footnote{Since this model only considers pairwise rankings, UQS and QS votes and credits yield the same result.}): three QS variants with different budgets, three LS variants with different budgets, a Likert scale survey, and a UQS condition. Since some participants worked on multiple survey instruments, $O_i$ captures the order in which the participant completed the survey $C_i$ to account for ordering effects. In addition, $P_i$ represents whether a participant’s pairwise ranking in an earlier survey aligned with the donation-based ranking, accounting for carryover effects. Lastly, $T_{1i}$ and $T_{2i}$ account for the topic-level effects of the two issues in comparison.

Given the complexity and nested structure of the data, we used a hierarchical Bayesian logistic regression model with non-centered parameterization~\cite{mcelreath2018statistical}. Hierarchical modeling enables partial pooling across different experimental conditions or topic pairings, which improves estimate robustness~\cite{mcelreath2018statistical}. 

We model the coefficients of each experimental variable ($\beta_{c}$, $\beta_{o}$, $\beta_{p}$, and $\beta_{t}$) using a hierarchical structure, drawing them from a normal distribution centered at a group-level mean $\mu_{\beta}$ and scaled by a group-level standard deviation $\sigma_{\beta}$. For example, the hierarchical structure of the coefficient $\beta_c$ for the survey condition variable $C_i$ is defined as: 

\vspace{-1em}

\begin{align}
    \label{eq:generic_non_center_hyper_C}
    \beta_c[C_i] &= \mu_{\beta_c} + \sigma_{\beta_c} \cdot \eta[C_i], \quad \eta[c_i] \sim \mathcal{N}(0, 1) \\
    \mu_{\beta_c} &\sim \mathcal{N}(0, 0.5), \quad   \sigma_c \sim \mathrm{Uniform}(0,1) 
\end{align}

Other coefficients follow the same structure, but some have different hyper-priors. Specifically, topic coefficients $\beta_{t}$ use a narrower hyper-prior $\mu_{\beta_t} \sim \mathcal{N}(0,0.25)$ to reflect a smaller expected effect.

\subsection{Pairwise Preference Intensity Model}
\label{sec:interval_measures}
The pairwise intensity model evaluates how effectively each survey instrument captures the magnitude of preference differences between options. We seek to model how the response difference between two options on a survey $\Delta_{\text{Survey}}$ \emph{predicts} the donation difference $\Delta_{\text{Donation}}$. Besides the eight survey conditions in the pairwise ranking model, we additionally analyze the number of credits spent on an option in QS (three budgets) and UQS (\Cref{tbl:experiment_cond}).

Comparing preference differences elicited via various survey instruments ($\Delta_{\text{Survey}}$) to donations ($\Delta_{\text{Donation}}$) is not trivial since some yielded continuous data while others were ordinal. Following the convention, we model $\Delta_{\text{Likert}}$ as ordinal data. Given the uncertainty about how participants accounted for the varying costs associated with QS votes, we treat $\Delta_{\text{QS Vote}}$ as ordinal as well. In contrast, LS votes increment with a consistent cost on a scale, therefore modeled as a continuous variable. UQS has no upper limit; hence, it is not ordinal. Finally, QS credits and monetary donations are continuous by nature.

Another challenge of this comparison is that raw differences from various instruments ($\Delta_{\text{Survey Raw}}$) and $\Delta_{\text{Donation Raw}}$ fall into varying data ranges. Thus, we apply the following data normalizations.

\paragraph{Normalize Continuous Survey Difference} We apply a variation of min-max scaling to project continuous $\Delta_{\text{Survey Raw}}$ onto the $[-1,1]$ interval\footnote{$\frac{x-min(x)}{(max(x) - min(x))}\times 2 - 1$}. For QS Credits and LS, we use the pre-defined bounds as the $min$ and $max$ in scaling. Since UQS lacks fixed bounds, we calculate the $min$ and $max$ for each participant based on the total votes and credits they used.

\paragraph{Normalize Ordinal Survey Difference} 
To enable direct comparison with continuous data, we project ordinal difference categories onto a latent continuous scale between 0 and 1, using cutpoints drawn from a Dirichlet-based model. Specifically, for each instrument, we derive $K$ discrete ordinal \emph{difference} categories. For example, vote difference in QS36 has 17 possible difference categories ($\Delta_{\text{QS36 Vote Raw}} = [-8, -7, ..., 7, 8]$)\footnote{Here, the largest vote difference possible with 36 credits occurs with 5 votes on option A and $-3$ votes on option B.}. We sample the second to $K$th elements of cutpoints \(\boldsymbol{\alpha}\) from a Dirichlet$(\mathbf{1}\cdot\delta)$ with $\delta=2$ as a weakly informative prior so that they sum to 1:

\begin{equation}
    \boldsymbol{\alpha}_{[2, ...,K]} \sim \mathrm{Dirichlet}\bigl(\mathbf{1}\cdot \delta\bigr), \text{where } \delta = 2
\end{equation}

The first cutpoint $\alpha_{1} = 0$. We then map ordinal $\Delta_{\text{Survey Raw}}=k$ to a latent continuous value between $[0,1]$ as $\Delta_{\text{Survey}}$:

\begin{equation}
    \text{For }\Delta_{\text{Survey Raw}}=k\text{, } \Delta_{\text{Survey}} = \sum_{j=1}^{k} \alpha_j.
\end{equation}

\paragraph{Normalize Donation Difference} Finally, we apply the same variation of min-max scaling to the donation differences of each participant based on their total donation amount. $\Delta_{\text{Donation}}$ ranges from $[-1, 1]$.

\textbf{Model Specification:} We model $\Delta_{\text{Donation}}$ as a Normal distribution:

\begin{equation}
    \label{eq:intensity_normal}
    \Delta_{\text{Donation}_i} \sim \mathcal{N}(\mu_{D_i}, \sigma_{D_i}).
\end{equation}

Since it is reasonable to expect that the variance in donation differs across experiment conditions, we make 
$\sigma_{D_i}$ condition-dependent: $\sigma_i=\beta_{\sigma}[C_i]$, where $\beta_{\sigma}[C_i]$ is drawn from the prior $\mathrm{Exponential}(1)$. $\mu_{D_i}$ is predicted by a linear regression of survey response difference $\Delta_{\text{Survey}_i}$, survey instrument $C_i$, survey order $O_i$, and topics $T_{1i}$, $T_{2i}$. 

\begin{equation}
    \label{eq:intensity_linpred}
    \mu_i
    =
    \beta_{\text{S}}[C_i] \cdot \Delta_{\text{Survey}_i}
    +
    \beta_{c}[C_i]
    +
    \beta_{o}[O_i]
    +
    \beta_{t}[T_{1i}]
    +
    \beta_{t}[T_{2i}].
\end{equation}

We model the slope of survey response differences $\beta_{\text{S}}$ for each survey condition with partial pooling and non-centered parameterization.

\begin{align}
    \beta_{\text{S}}[C_i]
    &=
    \mu_{\beta_{\text{vote}}}
    + \sigma_{\beta_{\text{vote}}} \cdot \eta_{\beta_{\text{vote}}}[C_i],
    \quad \eta[c_i] \sim \mathcal{N}(0, 1) \\
    \mu_{\beta_{\text{vote}}}
    &\sim
    \mathcal{N}(0,1),
    \quad
    \sigma_{\beta_{\text{vote}}}
    \sim
    \mathrm{Uniform}(0,1).
\end{align}

We model intercepts $\beta_{o}$ and $\beta_{t}$ in a similar way but with a hyper-prior of $\mu_{\beta} \sim \mathcal{N}(0,0.1)$. Finally, we sample the condition-based intercept $\beta_{c}$ from the prior $\mathcal{N}(0,0.2)$ without pooling. 
