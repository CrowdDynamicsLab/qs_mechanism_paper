\section{Discussion and Future Work}
\label{sec:discussion}
In this section, we answer our research questions and discuss our interpretation of this result. We then highlight recommendations to practitioners using QS for collective decision making and highlight future directions for QS.

\subsection{QS effectiveness and its mechanics}
To answer RQ1, which examines how effectively QS captures participant preferences compared to traditional methods, we applied detailed analysis across preference elicitation pairwise results for QS (votes and credit spent), Likert, UQS, and LS. Our analysis demonstrates that QS exhibits superior performance in preference elicitation, though with notable methodological distinctions. For ordinal preference prediction, QS shows a statistically significant but modest advantage over Likert scales. The substantive methodological benefit of QS becomes evident in preference intensity measurement—as the magnitude of preference differentials increases between options, QS maintains high fidelity in capturing these differentials, whereas alternative methodologies demonstrate decreasing accuracy.

Our results also address RQ2, which investigates how the budget constraint and quadratic cost individually and jointly affect preference elicitation quality. Our empirical findings substantiate that both the quadratic cost function and budget constraint constitute essential elements of the QS mechanism. The removal of either component (as implemented in Unlimited QS or Linear Survey) results in diminished performance relative to standard Likert scales, with Linear Survey demonstrating progressively deteriorating accuracy as the allocated budget increases. These results suggest significant caution is warranted when considering simplified variants of QS, as such modifications substantially compromise the mechanism's capacity to accurately reflect underlying preference structures, particularly in contexts where precise measurement of strong preference intensities is crucial.

% insert figure
\begin{figure}[h]
    \centering
    \includegraphics[width=\textwidth]{content/image/Predicted_Donation_Diff_Interval.pdf}
    \caption{Comparison of preference elicitation methods: QS, LS, UQS, and Linear Survey.}
    \label{fig:comparison}
\end{figure}

\subsection{Distortion in survey's `preference units'}
Among options, people have strong preferences between some pairs and less among others.~\Cref{fig:comparison} shows how different survey tools capture these pairwise differences. As discussed in Section~\ref{}, while QV credits remain reflective of individual's behavior, the stronger a participant treat a pairwise option across all options, the more LS response difference underestimates it. If we define a `preference unit' as each additional preference interval the participant expressed (i.e., an additional vote, or selecting the next preference interval), this result means that individuals `perceived' less preference increases per additional preference unit, hence the underestimation. This behavior mirrors psychophysics' Fechner laws and early CSS literature where researchers believe that human perceived stimuli decreases as the stimuli grew~\cite{}.

The results showed that credit spent remains a good estimate of individual behaviors. This is likely due to credit being a byproduct of additive votes. The quadratic nature of the cost corrected this decrease in perceived preference (stimuli), albeit slightly over estimate when differences are small. This overestimation comes from participants trying to express differences among the provided votes but the quadratic cost forced individuals to express slightly more, which can be observed where higher QS budgets' cost captures the smaller donation differences compared to the lesser budget by a bit. These participants have more room to demonstrate the differences.

Finally, votes are translated as the `presented' preference that a participant wanted to convey. Since the credits corrected the distortion individual's placed on the vote costs, the votes thus remains a stable reflection. 

The total budget on the other hand helped anchor how individuals treat each of the `preference units.' When we study how much individuals spend in UQS, more than half of the participants spent more than 54 credits, further distorting how individual's treat the perceived preference of each `preference unit'.

\subsection{Takeaway for practitioners}
1. Participants should be aware of what they want to measure.
2. If the goal is to create rough rankings for individuals, not aggregating, then perhapse Likert suffices.
3. However, when we step into the space of aggregation for collective decision making, surveyed preference would require QS
4. While there are little difference between different credit size, 36 remains the most stable nonstatistically difference in bayesian terms

\section{Limitations and Future work}
\label{sec:limitations}

\paragraph{`True preferences' and survey instruments}
It is important to acknowledge that donation behaviors aims to mirror tangible, monetary contributions that a person would do in real life, which offers real stakes that created such incentive compatible dichotomy. Thus, not all preferences are reflected through monetary means, and external factors might influence donation decisions.

\paragraph{Charities and government roles}
Charities is still different than government resources in societal issues

\paragraph{Future Work: mental model construction of QS respondents}
- if is not clear how individuals balance between perceived preference and presented preference. Prior literatue also highlighted this gap

\paragraph{Future Work: KS vs conjoint analysis vs QS}
- Other force choice tools such as  KS and Conjoint analysis is also worth investigating comparing QS.






% \subsection{Votes in QS are designed for aggregation}
% Emperically, we see 


% # Relationships Between Linear Voting, Donation Behavior, Budget in Quadratic Surveys, and Quadratic Survey Votes

% ## 1. Linear Voting
% - **Concept**: In linear voting systems, each vote is treated equally, and the cost of each additional vote remains constant.
% - **Outcome**: The difference in vote counts between options directly translates to perceived preference, assuming linear preference increments.
% - **Limitation**: It may not accurately reflect strong intensity differences, as stronger preferences are not weighted differently.

% ## 2. 

% ## 3. Budget in Quadratic Surveys
% - **Concept**: Participants have a limited budget to allocate votes, with costs increasing quadratically.
% - **Outcome**: Forces participants to weigh their choices more carefully, making stronger preferences more costly to express.
% - **Advantage**: Better reflects the strength of preferences as each additional vote requires more effort (cost).
% - **Limitation**: Interpretation of budgets requires understanding the quadratic nature and its impact on expressed preferences.

% ## 4. Quadratic Survey Votes
% - **Concept**: Uses a quadratic cost structure to allocate votes, where each additional vote costs more.
% - **Outcome**: Encapsulates stronger preferences through higher costs, making it harder for majority preferences to dominate without significant effort.
% - **Advantage**: Mitigates tyranny of the majority by making it costlier to exert higher influence.
% - **Interpretation**: The aggregate votes reflect group preferences, but do not linearly correspond to preference strength.

% ## Conclusion
% Understanding these relationships provides a comprehensive view of how different voting mechanisms capture preference intensity.
