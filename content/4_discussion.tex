\section{Discussion and Future Work}
\label{sec:discussion}

The results indicated a clear need for quadratic cost \emph{and} budget as essential components to QS.

\subsection{QS effectiveness and its mechanics}


\subsection{Distortion in preference units within surveys: diminishing returns}
The results from this study, using a more detailed approach to analysis reaffirmed prior literature's effectiveness of QS in eliciting individual preferences that aligns closer to participant behaviors. What is more surprising is the ineffectiveness of LS when compared to actual behaviors, especially for intensity measures. In this subsection, we try to explain why.

We interpret the results reflected the diminishing return of additional scales in non QS surveys. The more different one weighs two options, the more underestimation LS and Likert portrays. In intuitive terms, participants should have expressed even stronger preferences to the survey results. The results echo some early constant sum evaluations where experiments show distortions when the individuals allocated more points to specific options~\cite{}.

Because, QS vote costs scaled quadratically, and that participants are indeed weighing the tradeoffs between a provided budget, the budget becomes a good reflection of individual behaviors, as long as the survey designers believes in the assumption that donation is reflective of a person's real world behaviors. In other words, to quantify the actual differences between two options, a practitioner would derive individual's costs to compare it.

\subsection{Votes in QS are designed for aggregation}
The second contribution of this empirical study directly reflects the theoretical goals of Quadratic Voting which the quadratic mechanism aims to prevent tyranny of the majority. If participant's preference intensity aligned well with their donation behaviors, their `perceived' preferences of their response to options are accurate. These credits are then translated as votes as participant's `presented preferences,' which influences the outcome of the group.




% # Relationships Between Linear Voting, Donation Behavior, Budget in Quadratic Surveys, and Quadratic Survey Votes

% ## 1. Linear Voting
% - **Concept**: In linear voting systems, each vote is treated equally, and the cost of each additional vote remains constant.
% - **Outcome**: The difference in vote counts between options directly translates to perceived preference, assuming linear preference increments.
% - **Limitation**: It may not accurately reflect strong intensity differences, as stronger preferences are not weighted differently.

% ## 2. 

% ## 3. Budget in Quadratic Surveys
% - **Concept**: Participants have a limited budget to allocate votes, with costs increasing quadratically.
% - **Outcome**: Forces participants to weigh their choices more carefully, making stronger preferences more costly to express.
% - **Advantage**: Better reflects the strength of preferences as each additional vote requires more effort (cost).
% - **Limitation**: Interpretation of budgets requires understanding the quadratic nature and its impact on expressed preferences.

% ## 4. Quadratic Survey Votes
% - **Concept**: Uses a quadratic cost structure to allocate votes, where each additional vote costs more.
% - **Outcome**: Encapsulates stronger preferences through higher costs, making it harder for majority preferences to dominate without significant effort.
% - **Advantage**: Mitigates tyranny of the majority by making it costlier to exert higher influence.
% - **Interpretation**: The aggregate votes reflect group preferences, but do not linearly correspond to preference strength.

% ## Conclusion
% Understanding these relationships provides a comprehensive view of how different voting mechanisms capture preference intensity.


\section{Limitations and Future work}
\label{sec:limitations}

\paragraph{`True preferences' and survey instruments}
It is important to acknowledge that donation behaviors aims to mirror tangible, monetary contributions that a person would do in real life, which offers real stakes that created such incentive compatible dichotomy. Thus, not all preferences are reflected through monetary means, and external factors might influence donation decisions.

\paragraph{Charities and government roles}
