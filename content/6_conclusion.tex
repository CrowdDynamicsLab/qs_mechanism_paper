\section{Conclusion}
This work deepens our understanding of QS as a mechanism for eliciting richer signals in collective intelligence systems. By experimentally isolating the effects of the quadratic cost function and budget constraint, we compared QS with LS, UQS, and Likert scale surveys. Results show that removing either component weakens QS’s ability to capture accurate rankings and intensities, particularly as preference gaps grow.

These findings position QS as a promising instrument for collective decision-making in resource-constrained contexts where it is critical to capture not just what people prefer, but how strongly they prefer it. Future CI research should explore how people mentally model credit constraints, costs, and trade-offs when responding with QS, thereby guiding the design of more intuitive and effective tools for collective choice.

% This work advances the understanding of QS by examining how its embedded constraints shape expressed preferences and participant behavior. Through controlled comparisons with LS and UQS, we isolate the effects of the quadratic cost function and budget constraint on alignment with behavioral outcomes. QS performs well only when both components are present; removing either weakens its ability to capture accurate rankings and intensities, particularly as preference gaps grow. These findings support QS as a valuable tool for eliciting relative preferences in resource-constrained settings. They also highlight the need to better understand how individuals interpret votes, budgets, and effort when forming preferences. Future work should investigate these mental models to inform more effective and usable QS designs.