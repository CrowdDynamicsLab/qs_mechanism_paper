\section{Conclusion}
This study extends empirical studies on Quadratic surveys to understand empirically how its embedded constraints effected the elicited preferences with participant behaviors. We show that QS only aligns with their behavior when votes cost quadratically under a given constraint. We also recommend researchers who uses constant-sum surveys to consider QS as an alternative for situations where the survey designers wanted to understand relative preferences across multiple options in resource constraint scenarios. Finally, we call for future research to construct participant's mental model when constructing and expressing their preferences to understand how individuals cope with votes as preference and their willingness to spend for the votes.