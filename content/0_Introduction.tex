\section{Introduction}
% par 1: Introduction to the Problem
% Purpose: Define the problem and explain its significance.
%  What is the problem, what is the challenge, and why is it important

\begin{displayquote}
[T]he many, who are not as individuals excellent men, nevertheless can, when they have come together, be better than the few best people, not individually but collectively, just as feasts to which many contribute are better than feasts provided at one person's expense.

\begin{flushright}
-- Aristotle, \textit{Politics} III
\end{flushright}
\end{displayquote}

% collective intelligence requires good tools for individual prefernce construction and elicititation
Collective intelligence (CI) depends on effectively aggregating individual inputs to produce high-quality outcomes. This process relies on the assumption that the instrument captures individuals' complete set of expressed preferences and that the aggregation mechanism allows fair synthesis among these preferences into a coherent collective decision. Traditional survey methods---such as Likert scales---often fail to elicit both preference rankings and intensities, especially when limited resources must be distributed across multiple options. Likewise, one-person-one-vote (1p1v) systems can lead to the ``tyranny of the majority'' overlooking the strength of minority preferences. These limitations pose a significant challenge for CI systems that rely on nuanced preference data.

% QS is an recent alternative for good individual preference elicitation when compared to Likert.
Recent studies have explored Quadratic Surveys (QS) as an alternative to Likert scales for preference elicitation in resource constraint scenerios~\cite{chengCanShowWhat2021, quarfoot2017quadratic, cavaille2024cares}. Under QS, respondents use a fixed budget to ``purchase'' $k$ votes for each option, where the cost follows $k^2$. This quadratic cost function is intended to encourage participants to tradeoff carefully on how they distribute votes, revealing not only which options they favor but also how strongly they favor them. While QS appears promising, its reliance on both a fixed budget and a quadratic cost heightens cognitive load~\cite{chengOrganizeThenVote2025}, and raises a key question: could we achieve comparable expressiveness with a simpler linear scheme~\cite{quarfoot2017quadratic, chengCanShowWhat2021, cavaille2024cares} or by removing the budget constraint and rely on the quadratic cost function?


% ================================ %
% par 2: Approaches to Address the Challenges
% Purpose: Describe the existing approaches related to the problem.
% Key Questions:
%  - What are some broad approaches to addressing these challenges?
%  - Do not go into detail about related work but give an idea of the major themes in related work.

% prior research explore limited comparisons on the effectiveness of QS
Prior work has largely compared QS to Likert scales, finding that QS can capture preference intensities more effectively~\cite{quarfoot2017quadratic,naylor2017first,chengCanShowWhat2021,cavaille2024cares}. However, most studies do not disentangle QS's two central constraints: a fixed budget and a quadratic cost. Clarifying which element drives QS's alignment with participant behaviors offers an opportunity to streamline the mechanism without sacrificing effectiveness.


The linear variant of QS, which we term \emph{Linear Survey (LS)}, closely resembles Constant Sum Surveys (CSS)\cite{metfesselProposalQuantitativeReporting1947}. Despite its long history in marketing research and early validation, CSS has shown mixed results in more recent evaluations\cite{}. Although it is mathematically equivalent to LS, CSS frames the process differently by disallowing negative votes and often requiring respondents to use their entire budget. Further, there are no official guideline on how many options and credits designers can place on a CSS. This distinction motivates an empirical investigation to determine whether CSS---or a linear QS---can approximate the benefits of the full quadratic mechanism.

% ================================ %
% par 3: Your Proposal
% Purpose: Present your main ideas and proposed solution.
% Key Question:
%  - What are you proposing? Provide a sketch of the major ideas.

Thus, to better understand QS, we propose to disentangle its two components, budget and quadratic cost, and assess whether simpler mechanisms offer comparable results. We introduce two evaluation measures for ranking and rating that compares the different survey mechanisms, ``Unlimited'' QS (UQS for short), LV, Likert-scaled surveys (Likert for short), and QS with participant's donation behaviors. We aim to understand which survey instrument better reflects incentive compatible donation outcomes. Formally, our research questions are:

\begin{itemize}
    \item [\textbf{RQ1.}] How effectively does QS capture participant preferences in pairwise comparisons and preference intensities, relative to high-dimensional similarity measures (e.g., cosine similarity)?
    \item [\textbf{RQ2.}] How do the budget constraint and quadratic cost—core components of Quadratic Voting—individually and jointly affect the quality of elicited preferences?
\end{itemize}

% ================================ %
% par 4: Main Findings
% Purpose: Summarize the key findings from your work.
% Key Question:
%  - What are the main findings?

To investigate these questions, we recruited 202 MTurk participants using stratified sampling to approximate U.S. census demographics. Building on the software from~\citet{chengCanShowWhat2021}, we conduct four additional experiments that remove budget constraints or altered the cost function. Specifcially, participants completed either a UQS, or two versions of Linear Voting with different budgets (36, 108, or 324 credits) before completing a donation task. Together with open datasets on QS and Likert surveys~\cite{illinoisdatabankIDB-1928463}, we compare the effectiveness of these survey mechanisms using two Bayesian models to evaluate the ordinal and interval measures of the survey results and donation behaviors. Our results reveal that ...

% ================================ %
% par 5: Main Contributions
% Purpose: Identify and explain the primary contributions of your work.
% Key Structure:
%  1. Line 1: Identify your contribution—conceptualization, framework, interface, algorithm, etc.
%  2. Line 2: Contrast your contribution with prior work.
%  3. Line 3: Explain how you accomplished your contribution.
%  4. Line 4: Emphasize the impact of the contribution—why should anyone care?

\paragraph{Contributions}
This study reinforces the theoretical foundation of quadratic voting~\cite{lalley2016quadratic} while broadening its implications for the collective intelligence community in the following ways:
\begin{itemize}
    \item \textbf{Rigorous evaluation metrics.} We adopt a more nuanced approach to measuring alignment between survey responses and actual participant behaviors. Our results indicate that QS outperforms Likert scales and possibly CSS in resource-constrained scenarios.
    \item \textbf{Mechanics of QS.} By decomposing QS into its two components---budget and quadratic cost---we empirically confirm that both are necessary for capturing nuanced preference intensities.
    \item \textbf{Guidance for survey designers.} Our findings suggest that researchers relying on CSS may benefit from incorporating QS, which elicits both preference rankings and intensities more effectively.
\end{itemize}



% old tex
% comparing QS to Likert scales has primarily focused on evaluating its effectiveness in preference elicitation, but these studies often conflate rankings and ratings. QS is one of the few survey tools capable of eliciting both simultaneously, making it difficult to isolate how well each measure aligns with participant behavior independently. Additionally, while QS has been extensively compared to Likert scales, fewer studies have examined its relationship to a similar class of surveys—constant sum surveys (CSS)—which employ a linear cost function. Since QS's advantage over Likert scales is often attributed to the forced trade-offs introduced by its budget constraint, researchers have questioned whether a simpler linear cost structure could achieve similar results. Understanding whether a linear cost structure can replicate QS's effects is crucial for CI applications, where balancing cognitive effort and preference accuracy is key. 

% While recent investigations have explored the cognitive challenges QS imposes on participants, they do not clarify whether the quadratic cost function itself is necessary or whether a linear alternative could reduce complexity while maintaining effectiveness.