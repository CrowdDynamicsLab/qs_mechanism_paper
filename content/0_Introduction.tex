\section{Introduction}
% par 1: Introduction to the Problem
% Purpose: Define the problem and explain its significance.
%  What is the problem, what is the challenge, and why is it important

\begin{displayquote}
[T]he many, who are not as individuals excellent men, nevertheless can, when they have come together, be better than the few best people, not individually but collectively, just as feasts to which many contribute are better than feasts provided at one person's expense.

\begin{flushright}
-- Aristotle, \textit{Politics} III
\end{flushright}
\end{displayquote}

% % collective intelligence requires good tools for individual prefernce construction and elicititation
Collective decision-making processes, such as participatory budgeting (PB)\cite{desousasantosParticipatoryBudgetingPorto1998}, rely on the aggregation of individual preferences for the democratic allocation of limited resources. PB and similar community-driven decision-making processes benefit from eliciting preferences beyond simple rankings, as they allow for the capture of preference intensity. For instance, a scenario in which a minority holds strong preferences, even against a majority with only mild support, can substantially influence outcomes. Yet, traditional methods like Likert scales or public polls often fall short of revealing the subtle trade-offs among options required in high-stakes contexts~\cite{quarfoot2017quadratic,posner2017quadratic, krosnick1999survey}. Quadratic Surveys (QS) have been proposed as an alternative to Likert scale surveys, integrating a fixed budget with a quadratic cost function that nudges respondents to reveal trade-offs among many options~\cite{chengCanShowWhat2021, quarfoot2017quadratic, cavaille2024cares}. Although preliminary studies~\cite{chengCanShowWhat2021, cavaille2024cares} indicate that QS more effectively captures individual preferences than traditional Likert scales, the overall evidence remains sparse. Moreover, the complexity of QS’s quadratic cost mechanism has spurred interest in exploring potential simplifications, such as the adoption of a linear cost function~\cite{cavaille2024cares, chengOrganizeThenVote2025}.

% ================================ %
% par 2: Approaches to Address the Challenges
% Purpose: Describe the existing approaches related to the problem.
% Key Questions:
%  - What are some broad approaches to addressing these challenges?
%  - Do not go into detail about related work but give an idea of the major themes in related work.

Researchers attribute QS's effectiveness primarily to two components: a fixed budget constraint and a quadratic cost function. Prior research has predominantly compared QS with Likert scale surveys, highlighting its ability to capture realistic participant behaviors and clearly differentiate preference intensities when participants explicitly prioritize options due to imposed constraints~\cite{chengCanShowWhat2021, cavaille2024cares}. A closely related but simpler forced-choice approach, Constant Sum Surveys (CSS)~\footnote{CSS are also referred to as chip-allocation surveys, point-allocation surveys, fixed-sum surveys, and the budget pie method in the literature.}~\cite{metfesselProposalQuantitativeReporting1947}, has long utilized a linear constraint to require explicit prioritization among options. Despite the long-standing use of CSS and its linear budget constraint, prior research has not clearly isolated whether QS's effectiveness derives uniquely from its quadratic cost structure or if merely enforcing a fixed budget alone sufficiently generates the perceived trade-offs. Clarifying this distinction presents an opportunity to streamline QS, potentially reducing complexity without diminishing its effectiveness.

% ================================ %
% par 3: Your Proposal
% Purpose: Present your main ideas and proposed solution.
% Key Question:
%  - What are you proposing? Provide a sketch of the major ideas.

Thus, in this paper, we build upon open data from~\citet{chengCanShowWhat2021, illinoisdatabankIDB-1928463}, which provides a rich dataset and open-source software for modification, and introduce two additional survey variants designed to isolate the core components of QS. The first, which we call Unlimited QS (UQS), removes the budget constraint but retains the quadratic cost function. The second, Linear Survey (LS), retains the fixed budget but replaces the quadratic cost with a linear one. These two variants allow us to disentangle the individual effects of QS's budget constraint and cost function.

Furthermore, to better assess the different surveying methods, we develop two Bayesian models --- distinct from prior research --- to evaluate QS, Likert-scale surveys, UQS, and LS, grounded in the existing CSS evaluation literature and accounting for the mix of ordinal and continuous data. Existing CSS evaluation literature often compares survey responses at the level of pairwise options and behaviors, rather than treating each survey submission as a single unit—an approach better suited to capturing fine-grained preference differences~\cite{collewet2023preference, hauserIntensityMeasuresConsumer1980a}. Formally, we ask:

\begin{itemize}
    \item [\textbf{RQ1.}] How effectively does QS capture participant preferences in pairwise comparisons and preference intensities, relative to high-dimensional similarity measures (e.g., cosine similarity)?
    \item [\textbf{RQ2.}] How do the budget constraint and quadratic cost---core components of Quadratic Survey---individually and jointly affect the quality of elicited preferences?
\end{itemize}

% ================================ %
% par 4: Main Findings
% Purpose: Summarize the key findings from your work.
% Key Question:
%  - What are the main findings?

To investigate these questions, we recruited 202 MTurk participants using stratified sampling to approximate U.S. census demographics, utilizing modified open-source software from~\citet{chengCanShowWhat2021}. Participants completed one of three survey types: Unlimited QS (UQS), or one of two Linear Survey (LS) versions with different credit budgets (18, 54, or 162), followed by an incentive-compatible donation task. Together with open data from prior research, we developed two Bayesian models to evaluate survey effectiveness. The first assesses whether each method accurately captures the relative ranking of preferences between option pairs. The second examines whether larger differences in reported survey preferences correspond to greater behavioral intensity, offering an interval-based perspective.

Our findings show that, in terms of pairwise ordering, QS outperforms Likert scale surveys, while both UQS and LS underperform relative to Likert. For pairwise intensity differences, all methods perform similarly when the preference gap between two options is small. However, as the gap increases, QS—particularly its vote and credit-based measures—more reliably reflects behavioral intensity compared to other approaches. LS trails behind Likert under these conditions, with its performance deteriorating further as preference differences grow. These results highlight the importance of both the credit budget and the quadratic cost function in producing effective preference elicitation. Our findings reaffirm QS's ability to represent individual preferences in resource-constrained contexts and underscore the limitations of linear or unconstrained alternatives.

% ================================ %
% par 5: Main Contributions
% Purpose: Identify and explain the primary contributions of your work.
% Key Structure:
%  1. Line 1: Identify your contribution—conceptualization, framework, interface, algorithm, etc.
%  2. Line 2: Contrast your contribution with prior work.
%  3. Line 3: Explain how you accomplished your contribution.
%  4. Line 4: Emphasize the impact of the contribution—why should anyone care?

\paragraph{Contributions} This paper provides new empirical evidence about Quadratic Surveys (QS) by isolating and evaluating their two core components: fixed budgets and quadratic voting costs. Prior research established theoretical advantages of QS but did not clarify whether the budget constraint, the quadratic cost, or both are necessary to achieve these advantages. Through controlled laboratory experiments involving public resource allocation scenarios, incentive-compatible donation tasks, and Bayesian modeling, we measured how each component, budget constraint versus quadratic cost, influences the alignment between expressed preferences (votes) and real-world behavior (actual spending). Our findings reveal that removing either the quadratic cost function or the credit budget from QS significantly reduces its effectiveness in capturing both ordinal and interval differences between pairwise options. This critical distinction offers direct guidance for survey designers and practitioners working in public resource allocation and collective decision-making, helping them interpret QS outcomes more effectively and design preference elicitation tools that are both realistic and behaviorally aligned.

% old tex
% comparing QS to Likert scales has primarily focused on evaluating its effectiveness in preference elicitation, but these studies often conflate rankings and ratings. QS is one of the few survey tools capable of eliciting both simultaneously, making it difficult to isolate how well each measure aligns with participant behavior independently. Additionally, while QS has been extensively compared to Likert scales, fewer studies have examined its relationship to a similar class of surveys—constant sum surveys (CSS)—which employ a linear cost function. Since QS's advantage over Likert scales is often attributed to the forced trade-offs introduced by its budget constraint, researchers have questioned whether a simpler linear cost structure could achieve similar results. Understanding whether a linear cost structure can replicate QS's effects is crucial for CI applications, where balancing cognitive effort and preference accuracy is key. 

% While recent investigations have explored the cognitive challenges QS imposes on participants, they do not clarify whether the quadratic cost function itself is necessary or whether a linear alternative could reduce complexity while maintaining effectiveness.

%Some old tex belowPrior research investigating the effectiveness of QS has predominantly compared it with Likert scale surveys, emphasizing QS's strength in capturing realistic participant behaviors and clearly differentiating preference intensities when participants explicitly prioritize options due to imposed constraints~\cite{chengCanShowWhat2021, cavaille2024cares}. Despite this, most existing studies have not explicitly disentangled the individual effects of QS's two central constraints: a fixed budget and a quadratic cost function. Clarifying which component primarily drives QS's alignment with participant behaviors could enable researchers to streamline the mechanism, potentially reducing complexity and cognitive load without sacrificing effectiveness.

% Constant Sum Surveys (CSS), a simpler forced-choice method, utilize linear constraints requiring participants to allocate their entire budget positively across options~\cite{metfesselProposalQuantitativeReporting1947}. Although CSS has a long history in marketing research, recent evaluations have yielded mixed results. The absence of negative votes and the mandatory allocation of the entire budget distinguish CSS from QS. Investigating whether linear approaches such as CSS can approximate the effectiveness of QS's quadratic mechanism remains an important and under-explored research direction.