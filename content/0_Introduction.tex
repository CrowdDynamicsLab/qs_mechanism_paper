\section{Introduction}
% par 1: Introduction to the Problem
% Purpose: Define the problem and explain its significance.
%  What is the problem, what is the challenge, and why is it important

\begin{displayquote}
[T]he many, who are not as individuals excellent men, nevertheless can, when they have come together, be better than the few best people, not individually but collectively, just as feasts to which many contribute are better than feasts provided at one person's expense.

\begin{flushright}
-- Aristotle, \textit{Politics} III
\end{flushright}
\end{displayquote}

From renewable energy planning~\cite{fishkinConsultingPublicDeliberative2003} and ride-sharing regulation~\cite{hsiao2018vtaiwan} to corporate forecasting~\cite{cowgill2015corporate} and government budgeting~\cite{desousasantosParticipatoryBudgetingPorto1998}, both public initiatives and academic studies~\cite{woolley2010evidence, centola2022network, krosnick1999survey, weylPluralityFutureCollaborative2024} show that effective collective intelligence (CI) depends on integrating truthful, diverse, and rich preference signals. Conventional tools such as Likert scale surveys, public polls, and one-person-one-vote schemes reduce individual choices to surface-level tallies, thereby obscuring the intensity and trade-offs that lead to better outcomes~\cite{quarfoot2017quadratic,posner2017quadratic, krosnick1999survey}. Quadratic Survey (QS) addresses this limitation by assigning respondents a fixed credit budget and applying a quadratic cost to each vote, prompting survey respondents to express not only which options they support but also how strongly they care~\cite{chengCanShowWhat2021, quarfoot2017quadratic, cavaille2024cares}. QS can identify intense minority preferences when they outweigh mild majority support in resource-constrained scenarios such as public policy~\cite{chengCanShowWhat2021, quarfoot2017quadratic}, product design~\cite{chengCanShowWhat2021}, or co-housing communities~\cite{karpinskiPotentialQuadraticVoting2025}. However, the higher cognitive demands of completing QS~\cite{chengOrganizeThenVote2025} have spurred researchers~\cite{cavaille2024cares, chengCanShowWhat2021} to propose plausible simplifications, such as replacing quadratic costs with linear ones or removing budgets entirely, in order to reduce participant frustration. Yet, there remains limited empirical understanding of whether such changes preserve the features that make QS effective and of why QS works.

% ================================ %
% par 2: Approaches to Address the Challenges
% Purpose: Describe the existing approaches related to the problem.
% Key Questions:
%  - What are some broad approaches to addressing these challenges?
%  - Do not go into detail about related work but give an idea of the major themes in related work.

Researchers~\cite{chengCanShowWhat2021, cavaille2024cares} attribute QS's effectiveness primarily to two components: a fixed budget constraint and a quadratic cost function. Prior research has predominantly compared QS with Likert scale surveys, highlighting its effectiveness in capturing realistic participant behaviors and clearly distinguishing preference intensity across many options when participants must prioritize under constraints~\cite{chengCanShowWhat2021, cavaille2024cares}. A closely related but simpler forced-choice approach, constant sum survey (CSS)~\footnote{CSS is also referred to as chip-allocation survey, point-allocation survey, fixed-sum survey, and the budget pie method in the literature~\cite{harwoodUnderstandingImplicitExplicit2019, thomas2004using, robertsWeightApproximationsMultiattribute2002, zhuSelfestimationWeightParameter1991, mciverUsingBudgetPies1976, toepoelSmileysStarsHearts2019}.}~\cite{metfesselProposalQuantitativeReporting1947}, has long utilized a linear constraint to require explicit prioritization among options. Despite the long-standing use of CSS and its linear budget constraint, prior research has not clearly isolated whether QS's effectiveness derives uniquely from its quadratic cost structure or if merely enforcing a fixed budget alone sufficiently generates the perceived trade-offs. Clarifying this distinction presents an opportunity to streamline QS, potentially reducing complexity without diminishing its effectiveness.

% ================================ %
% par 3: Your Proposal
% Purpose: Present your main ideas and proposed solution.
% Key Question:
%  - What are you proposing? Provide a sketch of the major ideas.

We introduce two additional survey variants designed to isolate the core components of QS and evaluate them alongside QS and Likert scale survey responses reported in prior work~\cite{chengCanShowWhat2021}. The first, which we call Unlimited QS (UQS), removes the budget constraint but retains the quadratic cost function. The second, Linear Survey (LS), retains the fixed budget but replaces the quadratic cost with a linear one. These two variants allow us to disentangle the individual effects of QS's budget constraint and cost function. In addition, to our knowledge, no prior study has evaluated QS using pairwise comparisons of rankings and preference intensities, providing a more rigorous empirical lens~\cite{collewet2023preference}. Formally, we ask:

\begin{itemize}
    \item [\textbf{RQ1.}] How effectively does QS capture participant preferences in pairwise rankings and preference intensities compared to Likert scale survey?
    \item [\textbf{RQ2.}] How do the budget constraint and quadratic cost, as core components of QS, individually and jointly affect how well elicited preferences reflect participants' behavior?
\end{itemize}

% ================================ %
% par 4: Main Findings
% Purpose: Summarize the key findings from your work.
% Key Question:
%  - What are the main findings?

To investigate these questions, we recruited 202 MTurk participants using stratified sampling to approximate U.S. census demographics, using a modified version of open-source QS software described in prior work~\cite{chengCanShowWhat2021}. Participants completed either UQS or one of three LS versions with credit budgets of 18, 54, or 162. Each survey asked how the local government should allocate resources across a set of societal issues. Participants then completed an incentive-compatible donation task. Together with open data from prior research, we developed two Bayesian models to assess how well survey responses align with participants' actual behavior. The first assesses whether each method captures the same pairwise ranking of preferences as the donation task. The second examines whether larger differences in reported survey preferences correspond to greater behavioral intensity, offering an interval-based perspective.

Our findings show that, in terms of pairwise ranking, QS outperforms the Likert scale survey, while both UQS and LS underperform relative to the Likert scale survey. For pairwise intensity differences, all methods perform similarly when the preference gap between two options is small. However, as the gap increases, QS, both its vote and credit-based measures, more reliably reflects behavioral intensity compared to other approaches. UQS performs similarly to the Likert scale survey whereas LS trails behind the Likert scale survey under these conditions. Their performance deteriorates further as preference differences grow. These results highlight the importance of both the credit budget and the quadratic cost function in effective preference elicitation. Our findings reaffirm QS's ability to represent individual preferences in resource-constrained contexts and surface the limitations of linear or unconstrained alternatives.

% ================================ %
% par 5: Main Contributions
% Purpose: Identify and explain the primary contributions of your work.
% Key Structure:
%  1. Line 1: Identify your contribution—conceptualization, framework, interface, algorithm, etc.
%  2. Line 2: Contrast your contribution with prior work.
%  3. Line 3: Explain how you accomplished your contribution.
%  4. Line 4: Emphasize the impact of the contribution—why should anyone care?

This paper makes two contributions: an empirical analysis of QS's core mechanisms and a modeling approach for evaluating survey-behavior alignment.

\paragraph{Empirical Contribution: } 
This paper provides a more detailed empirical understanding of the Quadratic Survey mechanism by isolating and evaluating its two core components: fixed budgets and quadratic voting costs. Prior works~\cite{georgescuFixedbudgetMultipleissueQuadratic2024, eguia2019quadratic, quarfoot2017quadratic, chengCanShowWhat2021} established the theoretical and empirical grounds for QS's advantages but did not clarify whether the budget constraint, the quadratic cost, or both are necessary to achieve these advantages. We addressed this gap through controlled experiments framed around public resource allocation, using Bayesian modeling to examine how QS's budget and cost structures influence the alignment between stated survey preferences and participant donation behavior. Our findings reveal that removing either the quadratic cost or the credit budget weakens QS's ability to capture pairwise ranking and differences in preference intensity, suggesting that rather than simplifying QS through a linear cost function, future work should design interfaces to support survey respondents in expressing their preferences when answering QS.

% Contribution 2
\paragraph{Methodological Contribution:}
This paper introduces two Bayesian models to evaluate how different survey methods capture participants' preferences as reflected in their behavior. Prior evaluations have relied on submission-level or single-point behavioral comparisons, missing finer distinctions in preference structure. Our models evaluate both pairwise ranking and intensity by comparing stated pairwise preferences to donation behavior, drawing on common CSS pairwise comparison approaches~\cite{collewet2023preference, hauserIntensityMeasuresConsumer1980a}. This method enables more precise empirical evaluation of preference elicitation tools and informs future studies on QS design and validation.


%=======================END OF INTRO===============%

% old tex
% comparing QS to Likert scales has primarily focused on evaluating its effectiveness in preference elicitation, but these studies often conflate rankings and ratings. QS is one of the few survey tools capable of eliciting both simultaneously, making it difficult to isolate how well each measure aligns with participant behavior independently. Additionally, while QS has been extensively compared to Likert scales, fewer studies have examined its relationship to a similar class of surveys—constant sum surveys (CSS)—which employ a linear cost function. Since QS's advantage over Likert scales is often attributed to the forced trade-offs introduced by its budget constraint, researchers have questioned whether a simpler linear cost structure could achieve similar results. Understanding whether a linear cost structure can replicate QS's effects is crucial for CI applications, where balancing cognitive effort and preference accuracy is key. 

% While recent investigations have explored the cognitive challenges QS imposes on participants, they do not clarify whether the quadratic cost function itself is necessary or whether a linear alternative could reduce complexity while maintaining effectiveness.

%Some old tex belowPrior research investigating the effectiveness of QS has predominantly compared it with Likert scale surveys, emphasizing QS's strength in capturing realistic participant behaviors and clearly differentiating preference intensities when participants explicitly prioritize options due to imposed constraints~\cite{chengCanShowWhat2021, cavaille2024cares}. Despite this, most existing studies have not explicitly disentangled the individual effects of QS's two central constraints: a fixed budget and a quadratic cost function. Clarifying which component primarily drives QS's alignment with participant behaviors could enable researchers to streamline the mechanism, potentially reducing complexity and cognitive load without sacrificing effectiveness.

% Constant Sum Surveys (CSS), a simpler forced-choice method, utilize linear constraints requiring participants to allocate their entire budget positively across options~\cite{metfesselProposalQuantitativeReporting1947}. Although CSS has a long history in marketing research, recent evaluations have yielded mixed results. The absence of negative votes and the mandatory allocation of the entire budget distinguish CSS from QS. Investigating whether linear approaches such as CSS can approximate the effectiveness of QS's quadratic mechanism remains an important and under-explored research direction.