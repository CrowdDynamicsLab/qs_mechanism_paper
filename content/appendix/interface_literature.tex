\section{Voting Interface Breakdown}\label{apdx:relatedVoting}
In this section, we outline additional literature that informed this study. There are two sets of literature that we surveyed: Survey response format and voting interfaces.

\subsection{Survey response format}
Research in the marketing and research communities focusing on survey and questionnaire design, usability, and interactions examines the influence of presentation styles and `response format.'~\citet{weijtersExtremityHorizontalVertical2021} demonstrated that horizontal distances between options are more influential than vertical distances, with the latter recommended for reduced bias. Slider bars, which operate on a drag-and-drop principle, show lower mean scores and higher nonresponse rates compared to buttons, indicating they are more prone to bias and difficult to use. In contrast, visual analog scales that operate on a point-and-click principle perform better~\cite{toepoelSlidersVisualAnalogue2018}. These studies show how even small design changes can have a large impact on usability, highlighting the importance of designing interfaces that prioritize human-centered interaction rather than focusing solely on functionality.

\subsection{Voting Interfaces}
Compared to digital survey interfaces, voting interfaces are a specialized type of survey interface can significantly influence democratic processes~\cite{engstrom2020politics, chisnellDemocracyDesignProblem2016, civicdesignDesigningUsableBallots2015} and often have consequential impacts. Researchers believe that ill-designed voting interfaces We categorize these related works into three main categories detailed below:

\paragraph{Designs that shifted voter decisions: } For example, states without straight-party ticket voting~(where voters can select all candidates from one party through a single choice) exhibited higher rates of split-ticket voting~\cite{engstrom2020politics}. Another example from the Australian ballot showing incumbency advantages is where candidates are listed by the office they are running for, with no party labels or boxes.

\paragraph{Designs that influenced errors: } Butterfly ballots, an atypical design, may have influenced the outcome of the 2000 U.S. Presidential Election~\cite{wandButterflyDidIt2001}. It increased voter errors because voters could not correctly identify the punch hole on the ballot. Splitting contestants across columns increases the chance for voters to overvote~\cite{quesenberyOpinionGoodDesign2020}. On the other hand, \citet{everettElectronicVotingMachines2008} showed the use of incorporating physical voting behaviors, like lever voting, into graphical user interfaces increased satisfaction while maintaining efficiency and effectiveness.

\paragraph{Designs that incorporated technologies: } Other projects like the Caltech-MIT Voting Technology Project addresses accessibility challenges, resulting in innovations like EZ Ballot~\cite{leeUniversalDesignBallot2016}, Anywhere Ballot~\cite{summers2014making}, and Prime III~\cite{dawkinsPrimeIIIInnovative2009}. In addition,~\citet{gilbertAnomalyDetectionElectronic2013} investigated optimal touchpoints on voting interfaces, and~\citet{conradElectronicVotingEliminates2009} examined zoomable voting interfaces.

Response format literature and voting interfaces informed how interfaces significantly influence respondent behavior, decision accuracy, and cognitive load. These burdens are especially problematic for complex systems like QS, where high cognitive demands may deter researchers and users alike. Developing effective, human-centered interfaces for QS could enhance usability, reduce cognitive overload, and increase adoption in both research and practical applications.
