\subsection{Results Part I: Pairwise Preference Ranking Analysis}
\label{sec:result_1}

\textbf{Results interpretation: }To evaluate how well a survey tool reflects a participant's preference ranking between two causes, we calculate the posterior distribution of the probability that the pairwise preference ranking reflected through a survey tool aligns with that reflected in donation amounts. Furthermore, we compare survey tools' abilities to elicit accurate pairwise preference rankings using the odds ratio of the predicted odds of alignment between survey and donation preference rankings. For instance, an odds ratio of 2 between survey tools A and B means that the odds of participants expressing the same preference rankings in survey tool A and donations is twice of participants using survey tool B. We say that two survey tools differed significantly when the 94\% Highest Posterior Density Interval (HPDI) of the odds ratio's posterior distribution does not include the reference value of 1 (odds ratio = 1 means having the same odds). 

\textbf{QS with various budgets outperformed the Likert scale's ability to elicit preference rankings consistent with donations with a small effect size} (odds ratio mean = 1.65, 94\% HPDI = [1.55, 1.76])\footnote{Odds ratio = 1.68, 3.47, 6.71 corresponds to a small, medium, and large effect size, respectively~\cite{chen2010big}}. The model predicted that a participant's preference ranking in QS aligned with that in donations with a 70\% chance on average, higher than the 59\% average probability for the Likert scale. 

\textbf{When the budget from QS was removed, i.e., Unlimited QS, it performed worse than the Likert scale with a small effect size} (odds ratio mean = 0.59, 94\% HPDI = [0.56, 0.62]). Participants expressed consistent pairwise preference rankings with Unlimited QS and donations 46.2\% of the time on average (94\% HPDI = [35.0\%, 57.1\%]). 

\textbf{Linear Surveys (LS), a variation of QS with a linear instead of quadratic cost, was also less effective than the Likert scale with a small effect size} (odds ratio mean = 0.46, 94\% HPDI = [0.37, 0.55]). In addition, LS's performance worsened as its budget increased. The average predicted probability of consistent pairwise preference rankings between LS and donations was 43.9\%, 40.8\%, and 35.9\% for LS with a small, medium, and large budget, respectively.  