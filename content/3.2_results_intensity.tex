\subsection{Results Part II: Preference Intensity Analysis}
\label{sec:result_2}

\textbf{Results interpretation:} With the fitted intensity model, we calculate the posterior distribution of the mean of predicted donation differences, given an elicited preference difference intensity between any two causes. We construct three such distributions for each survey tool, one for a small, medium, and large preference difference elicited by the tool respectively (i.e., $\text{VoteDiff}_i$ = 0.19, 0.38, 0.57, corresponding to $median(\text{VoteDiff}_i)+k \times std(\text{VoteDiff}_i)$, where $k=1, 2, 3$). We then perform two comparison tasks using these posterior distributions. 

First, we evaluate if a distribution significantly differs from the ``perfect'' predicted donation difference. A normalized predicted donation difference between two causes is ``perfect'' when it equals the normalized difference between the preferences elicited by the survey tool. We conclude that a survey tool reflects a given preference difference intensity well when the distribution is \textit{not} significantly different from the ``perfect'' predicted donation difference $D_{ref}$, i.e., when the distribution's 94\% Highest Posterior Density Interval (HPDI) includes $D_{ref}$.

Second, we compare the posterior distributions of the mean of predicted donation differences between survey conditions for the same elicited preference difference intensity. Such comparisons provide insights into how a survey tool performs relative to another when at least one of them does not reflect a preference intensity well. For a pair of survey conditions, we construct the posterior distribution of Cohen's d to quantify the difference between the means of predicted donation. We report that a survey tool's ability to reflect a preference intensity differs from another when the 94\% HPDI of the Cohen's d distribution excludes zero. 

\textbf{When the preference difference was small, all tested survey tools reflected the difference well.} Among them, Likert, Unlimited QS, and LS results aligned best with donation differences (predicted mean = 0.14, 0.17, 0.16, respectively; perfect prediction = 0.19). When participants expressed a small difference in QS vote and credit between two causes, while their donation differences did not differ significantly from the ``perfect'' difference (0.19), they tended to enlarge the difference in donations (predicted mean = 0.29, 0.26, respectively). 

\textbf{As the intensity of preference difference expressed via the survey increased to medium to large size, only those elicited by QS (both vote and credit) were well-reflected in the corresponding donation differences.} For instance, when normalized QS vote and credit $\text{VoteDiff}_i = 0.38$, the model predicts an average donation difference of 0.38 (94\% HPDI for QS vote = [0.17, 0.61], for QS credit = [0.15, 0.58]). Results from QS aligned significantly better with donation results than those from the Likert scale with a medium to large effect size\footnote{Cohen's d = 0.2, 0.5, 0.8 corresponds to a small, medium, and large effect size, respectively}. Moreover, QS's advantage over the Likert scale increased with the intensity of preference difference. Using the donation prediction accuracy of  QS credit vs. Likert scale as an example, the mean Cohen's d increases from 0.71 to 0.99 when $\text{VoteDiff}_i$ changes from medium (Cohen's d 94\% HPDI = [0.61, 0.82]) to large (Cohen's d 94\% HPDI = [0.88, 1.10]).

While QS outperformed the Likert scale in its ability to predict donation difference for medium and large preference differences, \textbf{Unlimited QS (i.e., QS without a budget) predicted donation difference similarly to the Likert scale.} When $\text{VoteDiff}_i$ in Unlimited QS is large, it predicted smaller donation differences than the ``perfect'' amount (mean = 0.25, 94\% HPDI = [0.02, 0.46]; perfect prediction = 0.57). We observed a similar phenomenon with LS (i.e., QS without the quadratic cost). In fact, \textbf{LS's ability to predict donation differences for medium and large preference differences was worse than Unlimited QS with a small effect size} (when $\text{VoteDiff}_i = 0.57$, Cohen's d mean = 0.27, 94\% HPDI = [0.18, 0.37]).